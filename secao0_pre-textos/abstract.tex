The application of neural network models for selecting assets for investment portfolios using the Sharpe ratio has gained the attention of researchers. This study aims to apply neural networks in predicting the Sharpe ratio for investment portfolio selection. Through a literature review, the methods used in portfolio construction, optimization techniques, and neural network structures employed in this context are identified. The construction of prediction models requires the collection and preprocessing of historical data on financial assets and economic indicators. The portfolio selection process is divided into two stages: optimization based on the Sharpe ratio and portfolio selection using neural networks to predict the portfolio with the best performance in the next period. The performance evaluation is compared to the Bovespa index. The results show that the unconstrained model performs better in terms of optimization time and Sharpe ratio. Neural networks, although not surpassing the portfolios, demonstrate superior performance compared to the Ibovespa index. The LSTM+Bahdanau Attention neural network achieved the best performance. These findings contribute to the advancement of knowledge in the field of finance and highlight the potential of neural networks in investment portfolio selection.