% capa
    \onehalfspacing \begin{center}
        
        UNIVERSIDADE FEDERAL DE ITAJUBÁ\\
        CAMPUS DE ITABIRA \\
        MESTRADO PROFISSIONAL EM ENGENHARIA DE PRODUÇÃO 

    \end{center}
    \vspace{5cm}
    \center \AuthorName
    \vspace{5cm}
    \center \Title \textbf{:} \Subtitle
    \null\vfil
    \center \Location
    \center \Year
    \pagestyle{empty}
    \pagebreak

% folha de rosto
    \center \AuthorName
    \vspace{9cm}
    \center \Title \textbf{:} \Subtitle

    \vspace{4cm}
    \begin{changemargin}{8cm}{0cm} 
        \begin{fontsize}{10}{12} \selectfont
            Dissertação submetida ao Mestrado Profissional em Engenharia de Produção da Universidade Federal de Itajubá - campus de Itabira como requisito parcial para a obtenção do título de mestre em Engenharia de Produção - mestrado profissional. 
            
            Área de concentração: Engenharia de Produção

            Orientador: \Supervisor
        \end{fontsize}
    \end{changemargin}
    \null\vfil
    \center \Location
    \center \Year
    \pagebreak

% folha de identificação
    \center \AuthorName
    \vspace{5cm}
    \center \Title \textbf{:} \Subtitle

    \vspace{4cm}
    \begin{changemargin}{8cm}{0cm} 
        \begin{fontsize}{10}{12} \selectfont
            Dissertação submetida ao Mestrado Profissional em Engenharia de Produção da Universidade Federal de Itajubá - campus de Itabira como requisito parcial para a obtenção do título de mestre em Engenharia de Produção - mestrado profissional. 
            
            Área de concentração: Engenharia de Produção

            Banca examinadora de 14 de julho de 2023.
            
        \end{fontsize}
    \end{changemargin}

    \vspace{1cm}

    \textbf{Banca examinadora:}

    \vspace{1cm}    

    Prof. Henrique Duarte Carvalho, Dr. - UNIFEI
    
    \vspace{1cm}    
    Prof. Leonardo Albergaria Oliveira, Dr. - UNIFEI
    
    \vspace{1cm}    
    Prof. Michel Carlo Rodrigues Leles, Dr. - DTECH/UFSJ

    \null\vfil
    \center \Location
    \center \Year
    \pagebreak
    % \center

    % \null\vfil \vspace{12cm}

    % Ficha de identificação da obra

    % \fbox{\begin{minipage}{12cm} \center
    %     A ficha de identificação da obra é elaborada pelo próprio autor.

    %     Orientações em:

    %     XXXX
    %     \null\vfil \vspace{6cm}
    % \end{minipage}}
    % \pagebreak

% folha de aprovação
    % \center
    % \AuthorName
    % \vspace{1cm}

    % \Title \textbf{:} \Subtitle
    % \vspace{1cm}


    % O presente trabalho em nível de mestrado foi avaliado e aprovado por banca 
    
    % examinadora composta pelos seguintes membros:

    % \vspace{1cm}    
    % Prof.(a) XXXX Dr.(a)
    % Instituição XXXX
    
    % \vspace{1cm}    
    % Prof.(a) XXXX Dr.(a)
    % Instituição XXXX
    
    % \vspace{1cm}    
    % Prof.(a) XXXX Dr.(a)
    % Instituição XXXX
    
    % \vspace{1cm}    
    % \justifying \noindent \hspace{1.5cm} Certificamos que esta é a \textbf{versão original e final} do trabalho de conclusão que foi julgado adequado para obtenção do título de mestre em Engenharia de Produção - mestrado profissional obtido pelo Mestrado Profissional em Engenharia de Produção.

    % \center
    % \vspace{3cm}
    % \hspace{2cm} \hrulefill \hspace{2cm}

    % Coordenação do Mestrado Profissional em Engenharia de Produção

    % \vspace{1cm}
    % \hspace{2cm} \hrulefill \hspace{2cm}

    % \Supervisor,

    % Orientador(a)

    % \null\vfil
    % \Location, \Year.

    % \pagebreak

% % Dedicatória
%     \null\vfil
%     \begin{changemargin}{5cm}{0cm} 
%         Este trabalho é dedicado aos meus colegas de classe e ao meus queridos pais.
%     \end{changemargin}
%     \pagebreak

% Agradecimentos
    \center \fontsize{12}{14} \selectfont \section*{AGRADECIMENTOS}
    \justifying
    \hspace{1.5cm} Agradeço a Deus por me dar forças e sabedoria para concluir este trabalho. Agradeço à minha família, aos meus pais e em especial minha companheira, por todo o apoio e incentivo, que foram fundamentais para a realização deste trabalho. 

Agradeço ao meu orientador, Prof. Dr. Henrique Duarte Carvalho, pela paciência e compreensão, que diante dos momentos mais difíceis se mostrou um ser humano incrível, sempre disposto a ajudar. Além de demonstrar um profundo conhecimento e grande capacidade de transmitir sua sabedoria.

Aos professores do Programa de pós-graduação em Engenharia de Produção da Universidade Federal de Itajubá, pela formação e conhecimento adquirido.

A todos os meus amigos que, direta ou indiretamente, contribuíram para a realização deste trabalho.
    \pagebreak

% % Epígrafe
%     \null\vfil
%     \begin{changemargin}{5cm}{0cm} 
%         \justifying
%         \hspace{1.5cm} 
%         % A good portfolio is more than a long list of good stocks and bonds. It is a balanced whole, providing the investor with protections and opportunities with respect to a wide range of contingencies
%         " Um bom portfólio é mais do que uma longa lista de boas ações e títulos. É tudo equilibrado, proporcionando ao investidor proteções e oportunidades com relação a uma ampla gama de contingências." \textit{Harry Markowitz}
    
%     \end{changemargin}
%     \pagebreak

% Resumo
    \justifying
    \section*{\fontsize{12}{14} \selectfont RESUMO}
    Este trabalho tem como objetivo analisar a aplicação de redes neurais artificiais na previsão do índice Sharpe. Para isso, foram utilizados dados de 2010 a 2020 de 10 componentes do Ibovespa. Os resultados mostraram que a rede neural artificial apresentou um desempenho superior ao modelo de regressão linear múltipla, com um erro médio absoluto de 0,0001 e um erro médio quadrático de 0,0001. Além disso, a rede neural artificial apresentou um índice Sharpe médio de 0,0001, enquanto o modelo de regressão linear múltipla apresentou um índice Sharpe médio de 0,0001. Portanto, a rede neural artificial apresentou um desempenho superior ao modelo de regressão linear múltipla.

    \vspace{1cm}

    \noindent \textbf{Palavras-chave:} Redes Neurais, Índice Sharpe, Otimização de Carteiras, Mecanismo de Atenção

    \pagebreak

% Abstract
    \section*{\fontsize{12}{14} \selectfont \center ABSTRACT}
    \textit{This work aims to analyze the application of artificial neural networks in the prediction of the Sharpe index. For this, data from 2010 to 2020 of 10 components of the Ibovespa were used. The results showed that the artificial neural network presented a superior performance to the multiple linear regression model, with an average absolute error of 0.0001 and a mean squared error of 0.0001. In addition, the artificial neural network presented an average Sharpe index of 0.0001, while the multiple linear regression model presented an average Sharpe index of 0.0001. Therefore, the artificial neural network presented a superior performance to the multiple linear regression model.}

    \vspace{1cm}
    \noindent \textit{\textbf{Keywords:} artificial neural networks, prediction of the Sharpe index, multiple linear regression.}
    \pagebreak

% Lista de figuras
    {%
    \let\oldnumberline\numberline%
    \renewcommand{\numberline}[1]{\figurename~\oldnumberline{#1\enspace--}}%
    \listoffigures%
    }
    \thispagestyle{empty} % remove page number
    \pagebreak

% Lista de quadros
    {%
    \let\oldnumberline\numberline%
    % \renewcommand{\numberline}{Quadro~\oldnumberline}%
    \renewcommand{\numberline}[1]{Quadro~\oldnumberline{#1\enspace--}}
    \listof{quadro}{LISTA DE QUADROS}%
    }
    \pagebreak

% Lista de tabelas
    {%
    \let\oldnumberline\numberline%
    \renewcommand{\numberline}[1]{\tablename~\oldnumberline{#1\enspace--}}%
    \listoftables%
    }
    \thispagestyle{empty} % remove page number
    \pagebreak

% Lista de abreviaturas
    % \renewcommand{\glossarypreamble}{\glsfindwidesttoplevelname[\currentglossary]}
    \setglossarystyle{long}
    \setlength\LTleft{1.5cm}
    \setlength\LTright{0pt}
    \setlength\glsdescwidth{0.8\hsize}
    \doublespacing
    \printglossary[title=LISTA DE ABREVIATURAS E SIGLAS , type=\acronymtype]
    \setcounter{table}{0}
    % \pagebreak

% Lista de símbolos
    % \listofsymbols
    % \pagebreak

% Sumário
    \doublespacing
    \renewcommand{\contentsname}{}
    \tableofcontents
    \addtocontents{toc}{\protect\newpage\centering{\textbf{\uppercase{Sumário}}}} % força o sumário a começar em uma nova página
    \onehalfspacing
    \thispagestyle{empty} % remove page number
    \pagebreak

% % Página em branco
%     \thispagestyle{empty}
%     \null\vfil
%     \pagebreak

\pagestyle{myheadings} % numbering top right