% capa
    \onehalfspacing \begin{center}
        UNIVERSIDADE FEDERAL DE ITAJUBÁ\\
        CAMPUS DE ITABIRA \\
        MESTRADO PROFISSIONAL EM ENGENHARIA DE PRODUÇÃO 
    \end{center}
    \vspace{5cm}
    \center \AuthorName
    \vspace{5cm}
    \center \Title \textbf{:} \Subtitle
    \null\vfil
    \center \Location
    \center \Year
    \pagestyle{empty}
    \pagebreak

% folha de rosto
    \center \AuthorName
    \vspace{9cm}
    \center \Title \textbf{:} \Subtitle

    \vspace{4cm}
    \begin{changemargin}{8cm}{0cm} 
        \begin{fontsize}{10}{12} \selectfont
            Dissertação submetida ao Mestrado Profissional em Engenharia de Produção da Universidade Federal de Itajubá - campus de Itabira como requisito parcial para a obtenção do título de mestre em Engenharia de Produção - mestrado profissional. 
            
            Área de concentração: Engenharia de Produção

            Orientador: \Supervisor
        \end{fontsize}
    \end{changemargin}
    \null\vfil
    \center \Location
    \center \Year
    \pagebreak

% folha de identificação
    \center \AuthorName
    \vspace{5cm}
    \center \Title \textbf{:} \Subtitle

    \vspace{4cm}
    \begin{changemargin}{8cm}{0cm} 
        \begin{fontsize}{10}{12} \selectfont
            Dissertação submetida ao Mestrado Profissional em Engenharia de Produção da Universidade Federal de Itajubá - campus de Itabira como requisito parcial para a obtenção do título de mestre em Engenharia de Produção - mestrado profissional. 
            
            Área de concentração: Engenharia de Produção

            Banca examinadora de 14 de julho de 2023.
            
        \end{fontsize}
    \end{changemargin}

    \vspace{1cm}

    \textbf{Banca examinadora:}

    \vspace{1cm}    

    Prof. Henrique Duarte Carvalho, Dr. - UNIFEI
    
    \vspace{1cm}    
    Prof. Leonardo Albergaria Oliveira, Dr. - UNIFEI
    
    \vspace{1cm}    
    Prof. Michel Carlo Rodrigues Leles, Dr. - DTECH/UFSJ

    \null\vfil
    \center \Location
    \center \Year
    \pagebreak
    % \center

    % \null\vfil \vspace{12cm}

    % Ficha de identificação da obra

    % \fbox{\begin{minipage}{12cm} \center
    %     A ficha de identificação da obra é elaborada pelo próprio autor.

    %     Orientações em:

    %     XXXX
    %     \null\vfil \vspace{6cm}
    % \end{minipage}}
    % \pagebreak

% folha de aprovação
    % \center
    % \AuthorName
    % \vspace{1cm}

    % \Title \textbf{:} \Subtitle
    % \vspace{1cm}


    % O presente trabalho em nível de mestrado foi avaliado e aprovado por banca 
    
    % examinadora composta pelos seguintes membros:

    % \vspace{1cm}    
    % Prof.(a) XXXX Dr.(a)
    % Instituição XXXX
    
    % \vspace{1cm}    
    % Prof.(a) XXXX Dr.(a)
    % Instituição XXXX
    
    % \vspace{1cm}    
    % Prof.(a) XXXX Dr.(a)
    % Instituição XXXX
    
    % \vspace{1cm}    
    % \justifying \noindent \hspace{1.5cm} Certificamos que esta é a \textbf{versão original e final} do trabalho de conclusão que foi julgado adequado para obtenção do título de mestre em Engenharia de Produção - mestrado profissional obtido pelo Mestrado Profissional em Engenharia de Produção.

    % \center
    % \vspace{3cm}
    % \hspace{2cm} \hrulefill \hspace{2cm}

    % Coordenação do Mestrado Profissional em Engenharia de Produção

    % \vspace{1cm}
    % \hspace{2cm} \hrulefill \hspace{2cm}

    % \Supervisor,

    % Orientador(a)

    % \null\vfil
    % \Location, \Year.

    % \pagebreak

% % Dedicatória
%     \null\vfil
%     \begin{changemargin}{5cm}{0cm} 
%         Este trabalho é dedicado aos meus queridos pais e minha companheira. Também a Harry Markowitz, por ser o fruto de seu trabalho que me inspirou a realizar este trabalho.
%     \end{changemargin}
%     \pagebreak

% Agradecimentos
    \center \fontsize{12}{14} \selectfont \section*{AGRADECIMENTOS}
    \justifying
    \hspace{1.5cm} Agradeço a Deus por me dar forças e sabedoria para concluir este trabalho. Agradeço a minha família, em especial aos meus pais, por todo o apoio e incentivo. Agradeço ao meu orientador, Prof. Dr. Henrique Duarte Carvalho, pela paciência e dedicação. Agradeço aos meus colegas de classe, pela ajuda e incentivo. Agradeço a todos que de alguma forma contribuíram para a realização deste trabalho.
    \pagebreak

% % Epígrafe
%     \null\vfil
%     \begin{changemargin}{5cm}{0cm} 
%         \justifying
%         \hspace{1.5cm} 
%         % A good portfolio is more than a long list of good stocks and bonds. It is a balanced whole, providing the investor with protections and opportunities with respect to a wide range of contingencies
%         " Um bom portfólio é mais do que uma longa lista de boas ações e títulos. É tudo equilibrado, proporcionando ao investidor proteções e oportunidades com relação a uma ampla gama de contingências." \textit{Harry Markowitz}
    
%     \end{changemargin}
%     \pagebreak

% Resumo
    \justifying
    \section*{\fontsize{12}{14} \selectfont RESUMO}
    Este trabalho tem como objetivo analisar a aplicação de redes neurais artificiais na previsão do índice Sharpe. Para isso, foram utilizados dados de 2010 a 2020 de 10 componentes do Ibovespa. Os resultados mostraram que a rede neural artificial apresentou um desempenho superior ao modelo de regressão linear múltipla, com um erro médio absoluto de 0,0001 e um erro médio quadrático de 0,0001. Além disso, a rede neural artificial apresentou um índice Sharpe médio de 0,0001, enquanto o modelo de regressão linear múltipla apresentou um índice Sharpe médio de 0,0001. Portanto, a rede neural artificial apresentou um desempenho superior ao modelo de regressão linear múltipla.

    \vspace{1cm}

    \noindent \textbf{Palavras-chave:} Redes Neurais, Índice Sharpe, Otimização de Carteiras

    \pagebreak

% Abstract
    \section*{\fontsize{12}{14} \selectfont \center ABSTRACT}
    \textit{The application of neural network models for selecting assets for investment portfolios using the Sharpe ratio has gained the attention of researchers. This study aims to apply neural networks in predicting the Sharpe ratio for investment portfolio selection. Through a literature review, the methods used in portfolio construction, optimization techniques, and neural network structures employed in this context are identified. The construction of prediction models requires the collection and preprocessing of historical data on financial assets and economic indicators. The portfolio selection process is divided into two stages: optimization based on the Sharpe ratio and portfolio selection using neural networks to predict the portfolio with the best performance in the next period. The performance evaluation is compared to the Bovespa index. The results show that the unconstrained model performs better in terms of optimization time and Sharpe ratio. Neural networks, although not surpassing the portfolios, demonstrate superior performance compared to the Ibovespa index. The LSTM+Bahdanau Attention neural network achieved the best performance. These findings contribute to the advancement of knowledge in the field of finance and highlight the potential of neural networks in investment portfolio selection.}

    \vspace{1cm}
    \noindent \textit{\textbf{Keywords:} Neural Networks, Sharpe Ratio, Portfolio Optimization, Attention Mechanism}
    \pagebreak

% Lista de figuras
    {%
    \let\oldnumberline\numberline%
    \renewcommand{\numberline}{\figurename~\oldnumberline}%
    \listoffigures%
    }
    \thispagestyle{empty} % remove page number
    \pagebreak

% Lista de quadros
    {%
    \let\oldnumberline\numberline%
    \renewcommand{\numberline}{Quadro~\oldnumberline}%
    \listof{quadro}{LISTA DE QUADROS}%
    }
    \pagebreak

% Lista de tabelas
    {%
    \let\oldnumberline\numberline%
    \renewcommand{\numberline}{\tablename~\oldnumberline}%
    \listoftables%
    }
    \thispagestyle{empty} % remove page number
    \pagebreak

% Lista de abreviaturas
    \printglossary[title=LISTA DE ABREVIATURAS E SIGLAS , type=\acronymtype]
    \pagebreak

% Lista de símbolos
    % \listofsymbols
    \pagebreak

% Sumário
    \doublespacing
    \renewcommand{\contentsname}{}
    \tableofcontents
    \addtocontents{toc}{\protect\newpage\centering{\textbf{\uppercase{Sumário}}}} % força o sumário a começar em uma nova página
    \onehalfspacing
    \thispagestyle{empty} % remove page number
    \pagebreak

% Página em branco
    \thispagestyle{empty}
    \null\vfil
    \pagebreak

\pagestyle{myheadings} % numbering top right