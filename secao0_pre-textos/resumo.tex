Este trabalho tem como objetivo analisar a aplicação de redes neurais artificiais na previsão do índice Sharpe. Para isso, foram utilizados dados de 2010 a 2020 de 10 componentes do Ibovespa. Os resultados mostraram que a rede neural artificial apresentou um desempenho superior ao modelo de regressão linear múltipla, com um erro médio absoluto de 0,0001 e um erro médio quadrático de 0,0001. Além disso, a rede neural artificial apresentou um índice Sharpe médio de 0,0001, enquanto o modelo de regressão linear múltipla apresentou um índice Sharpe médio de 0,0001. Portanto, a rede neural artificial apresentou um desempenho superior ao modelo de regressão linear múltipla.