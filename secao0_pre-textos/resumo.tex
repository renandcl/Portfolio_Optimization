A aplicação de modelos de redes neurais com intuito de selecionar ativos para carteiras de investimentos com o uso do índice Sharpe é uma área que tem atraído a atenção de pesquisadores. Este estudo tem como objetivo aplicar redes neurais na previsão do índice Sharpe para seleção de carteiras de investimento. Por meio de uma revisão bibliográfica, são identificados os métodos utilizados na construção de carteiras, assim como as técnicas de otimização e as estruturas de redes neurais empregadas nesse contexto. A construção dos modelos de previsão requer a coleta e o tratamento de dados históricos de ativos financeiros e dados econômicos. O processo de seleção de carteiras é dividido em duas etapas: a otimização baseada no índice Sharpe e a seleção da carteira utilizando redes neurais para previsão da carteira com melhor desempenho no período seguinte. A avaliação do desempenho é realizada em comparação com o índice Bovespa. Os resultados mostram que o modelo sem restrições apresenta melhor desempenho em termos de tempo de otimização e índice Sharpe. As redes neurais, embora não superem as carteiras, demonstram desempenho superior ao Ibovespa. A rede neural LSTM+Atenção de Bahdanau obteve o melhor desempenho. Essas descobertas contribuem para o avanço do conhecimento na área de finanças e destacam o potencial das redes neurais na seleção de carteiras de investimento.