\usepackage[brazil]{babel} % section and appendix headings in Portuguese
\usepackage{setspace} % line spacing
\usepackage{times} % font times new roman
\usepackage[bottom=2cm,top=3cm,left=3cm,right=2cm]{geometry} % margins ABNT
\usepackage[document]{ragged2e} % justified text
\usepackage{amsmath,amssymb} % math symbols
\usepackage{graphicx} % figures
\usepackage{float} % position of contents right after text
\usepackage{subcaption}  % subcaptions for figures, tables etc
\usepackage{caption} % captions for figures, tables etc
\usepackage[hidelinks]{hyperref} % hyperlinks for references, figures and tables without highlight box
\usepackage{multirow} % multirow tables
\usepackage{algorithm,algcompatible} % algorithms
\usepackage[automake,nonumberlist,acronym]{glossaries} % glossary and acronyms list
\usepackage[alf,abnt-repeated-title-omit=yes,abnt-emphasize=bf,abnt-full-initials=no,abnt-doi=doi]{abntex2cite}
\citebrackets() % parenthesis brackets for citations

\makeglossaries % create glossary and acronyms list

\setlength\parindent{0pt} % no indentation
\newcommand\ipar{\par\noindent\hspace{1.5cm}} % paragraphs indentation

\setcounter{secnumdepth}{5} % section numbering depth
\setcounter{tocdepth}{3} % table of contents depth

\def\changemargin#1#2{\list{}{\rightmargin#2\leftmargin#1}\item[]} % paragraph margin adjustment
\let\endchangemargin=\endlist % paragraph margin adjustment

\usepackage{titlesec} % section headings format
\titleformat{name=\section,numberless}{\filcenter\normalfont\normalsize\bfseries\uppercase}{}{0em}{}
\titleformat{\section}{\normalfont\normalsize\bfseries\uppercase}{\thesection}{1em}{}
\titleformat{\subsection}{\normalfont\normalsize\uppercase}{\thesubsection}{1em}{}
\titleformat{\subsubsection}{\normalfont\normalsize\bfseries}{\thesubsubsection}{1em}{}
\titleformat{\paragraph}{\normalfont\normalsize\itshape}{\theparagraph}{1em}{}
\titleformat{\subparagraph}{\normalfont\normalsize}{\thesubparagraph}{1em}{}


\DeclareCaptionType[fileext=loc]{quadro}[Quadro][Lista de Quadros] % create float type for frames, similar to tables

\usepackage{tocbasic} % table of contents format

\DeclareTOCStyleEntry[
  entrynumberformat=\entrynumberwithprefix{\figurename},
  dynnumwidth
%   numsep=1em
]{tocline}{figure}

\DeclareTOCStyleEntry[
    entrynumberformat=\entrynumberwithprefix{\tablename},
    dynnumwidth
    %   numsep=1em
]{tocline}{table}
        
\newcommand\entrynumberwithprefix[2]{#1\enspace#2-\enspace}
\captionsetup{labelsep=endash} % caption label separator '-'

% \usepackage{showframe} % just for the example
\usepackage{tocloft} % table of contents format

\renewcommand{\cfttoctitlefont}{\hspace*{\fill}\normalsize\bfseries\MakeUppercase}
\renewcommand{\cftaftertoctitle}{\hspace*{\fill}}
\renewcommand{\cftlottitlefont}{\hspace*{\fill}\normalsize\bfseries\MakeUppercase}
\renewcommand{\cftafterlottitle}{\hspace*{\fill}}
\renewcommand{\cftloftitlefont}{\hspace*{\fill}\normalsize\bfseries\MakeUppercase}
\renewcommand{\cftafterloftitle}{\hspace*{\fill}}


\usepackage{titletoc} % table of contents alignment
\dottedcontents{section}[2cm]{\bfseries}{2cm}{0.5pc}
\dottedcontents{subsection}[2cm]{}{2cm}{0.5pc}
\dottedcontents{subsubsection}[2cm]{\bfseries}{2cm}{0.5pc}
\dottedcontents{paragraph}[2cm]{\itshape}{2cm}{0.5pc}
\dottedcontents{subparagraph}[2cm]{}{2cm}{0.5pc}
\renewcommand{\contentsname}{\centering Contents}



\usepackage{listingsutf8}
\usepackage[dvipsnames]{xcolor}
\usepackage{empheq} 

\newcommand{\boxedeq}[2]{\begin{empheq}[box={\fboxsep=6pt\fbox}]{align}\notag#2\end{empheq}}

\lstdefinestyle{code_style}{
    inputencoding=utf8/latin1,
    extendedchars=true,
    breaklines=true,
    backgroundcolor=\color{white},   
    commentstyle=\color{gray},
    keywordstyle=\color{Green},
    numberstyle=\tiny\color{Gray},
    stringstyle=\color{Black},
    basicstyle=\ttfamily\footnotesize\color{Blue},
    breakatwhitespace=false,         
    breaklines=true,                 
    captionpos=t,                    
    keepspaces=true,                 
    numbers=left,                    
    numbersep=5pt,                  
    showspaces=false,                
    showstringspaces=false,
    showtabs=false,                  
    tabsize=2,
    frame = shadowbox,
    rulecolor=\color{Black},
    literate=%
    {á}{{\'a}}1
    {í}{{\'i}}1
    {é}{{\'e}}1
    {ý}{{\'y}}1
    {ú}{{\'u}}1
    {ó}{{\'o}}1
    {ě}{{\v{e}}}1
    {š}{{\v{s}}}1
    {č}{{\v{c}}}1
    {ř}{{\v{r}}}1
    {ž}{{\v{z}}}1
    {ď}{{\v{d}}}1
    {ť}{{\v{t}}}1
    {ň}{{\v{n}}}1                
    {ů}{{\r{u}}}1
    {Á}{{\'A}}1
    {Í}{{\'I}}1
    {É}{{\'E}}1
    {Ý}{{\'Y}}1
    {Ú}{{\'U}}1
    {Ó}{{\'O}}1
    {Ě}{{\v{E}}}1
    {Š}{{\v{S}}}1
    {Č}{{\v{C}}}1
    {Ř}{{\v{R}}}1
    {Ž}{{\v{Z}}}1
    {Ď}{{\v{D}}}1
    {Ť}{{\v{T}}}1
    {Ň}{{\v{N}}}1                
    {Ů}{{\r{U}}}1  
    {á}{{\'a}}1
    {à}{{\`a}}1
    {ã}{{\~a}}1
    {é}{{\'e}}1
    {ê}{{\^e}}1
    {í}{{\'i}}1
    {ó}{{\'o}}1
    {õ}{{\~o}}1
    {ú}{{\'u}}1
    {ü}{{\"u}}1
    {ç}{{\c{c}}}1  
}
\renewcommand{\lstlistingname}{Código} % Listing->Code
\renewcommand{\lstlistlistingname}{Lista de \lstlistingname s}% List of Listings -> List of Algorithms
\lstset{style=code_style}
