%-------------------------------------------------------
\section{Conclusão}
%-------------------------------------------------------


\begin{frame}{Conclusão}

    \Large

    O modelo sem restrições obteve melhor desempenho em tempo de otimização e índice Sharpe, enquanto o modelo com restrições e heurísticas teve desempenho inferior, exceto o modelo com início aleatório.

    As redes neurais não superaram o desempenho das carteiras de investimento, mas superaram o índice Ibovespa. A rede neural LSTM+Atenção de Bahdanau teve o melhor desempenho, embora abaixo da carteira de 60 dias. A rede neural GRU+Atenção de Bahdanau teve o pior desempenho, ficando abaixo do índice Ibovespa em alguns momentos.

\end{frame}
\note{Específicas}


\begin{frame}{Conclusão}

    \Large

    Aplicou-se redes neurais na previsão do índice Sharpe para seleção de carteiras de investimento. Identificado referências relevantes e as técnicas utilizadas. Elaborado modelo de processamento de dados históricos e econômicos. Construídos modelos de otimização e de previsão com redes neurais. Avaliado o desempenho dos modelos.

    Essas descobertas contribuem para o avanço do conhecimento na área de finanças e destacam o potencial das abordagens baseadas em redes neurais na seleção de carteiras de investimento.

    Como sugestões para trabalhos futuros, pode-se a aplicação de dados financeiros de análise técnica no treinamento de redes neurais e também a aplicação de redes neurais em outros modelos de otimização, como o CVaR.


\end{frame}
\note{Gerais e futuros trabalhos}