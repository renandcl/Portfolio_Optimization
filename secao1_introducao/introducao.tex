\begin{section}{INTRODUÇÃO}

    \begin{subsection}{CONTEXTUALIZAÇÃO}

        \ipar O mercado financeiro gerencia trilhões de dólares com uso da teoria moderna de portfólios elaborada pelo laureado do prêmio Nobel de economia Harry Markowitz \cite{sethi2021nobel}. As discussões sobre a teoria foram aprofundadas pelo também laureado William Sharpe, que aborda a estratégia de maximização de uma carteira de investimentos, considerando tanto o risco quanto o retorno. 
        
        \ipar A intuição do investidor seria que a máxima diversificação para um retorno esperado da carteira geraria um portfólio de mínima variância, isto é, com menor risco. Contudo, esta hipótese é rejeitada segundo a teoria de \citeonline{markowitz1952portfolio}, pois existe uma combinação ideal de ativos que compõem uma carteira de forma eficiente maximizando o retorno com o menor nível de risco. Portanto, a cada retorno esperado há uma combinação eficiente dos ativos que gera uma carteira de mínima variância, assim formando uma fronteira eficiente de combinações de ativos.

        \ipar A escolha da carteira eficiente é feita com base na relação entre risco e retorno. Segundo \citeonline{sharpe1964capital}, os investidores exigem um prêmio de retorno para assumir riscos adicionais. Desta maneira, a preferência dá-se pela aceitação ao risco, em que para opções de investimentos com o mesmo retorno esperado, o investidor escolherá a opção com menor risco \cite{lintner1965valuation}.
        
        \ipar Este prêmio por unidade de risco é conhecido como o índice Sharpe \cite{sharpe1994sharpe}. O termo foi apresentado por \citeonline{treynor1973use} em reconhecimento as contribuições feitas na avaliação de desempenho de fundos feita por \citeonline{sharpe1966mutual}, e atualmente é uma das medidas mais utilizadas para avaliação de desempenho de carteiras de investimento.

        \ipar Considerando que existe a opção para o investidor realizar empréstimo ou tomar emprestado a uma taxa livre de risco, é possível construir uma carteira eficiente que combina o empréstimo com todos os investimentos disponíveis no mercado. Esta combinação forma uma linha reta que tangencia a fronteira eficiente de ativos. Esta linha é denominada linha do mercado de capitais, e a carteira eficiente que tangencia a linha é chamada de carteira de mercado, que apresenta o maior prêmio por unidade risco \cite{sharpe1964capital}.

        \ipar Uma estratégia de construção de carteira de investimentos é a otimização do índice Sharpe, que consiste em maximizar o índice Sharpe para uma carteira de ativos para obter a carteira de mercado \cite{maree2022balancing}. Para a otimização da carteira é necessário estimar o retorno e o risco da carteira. A estimativa do retorno esperado é feita com base em dados históricos, e o risco é estimado com base na matriz de correlação dos ativos e a volatilidade de cada ativo.

        \ipar Modelos de otimização também combinam previsão das séries temporais para estimar o retorno esperado da carteira. Modelos tradicionais de previsão de séries temporais, aplicam métodos estatísticos que consideram as séries como modelos lineares \cite{zhou2023twostage}. Dentre estes há os modelos de médias móveis auto-regressivas integradas (\acrshort{ARIMA}, do inglês \textit{Autoregressive Integrated Moving Average}), e heterocedasticidade condicional autoregressiva generalizada (\acrshort{GARCH}, do inglês \textit{Generalized Autoregressive Conditional Heterocedasticity}). 
        
        \ipar Entretanto, séries temporais de ativos financeiros como ações, apresentam uma dinâmica de processo não linear. Há diversas abordagens para capturar dados não lineares, dentre estes há as redes neurais artificiais, modelos de redes neurais profundas, ou a rede neural recorrente que se provaram como ferramentas válidas para aplicações para dados de séries temporais multivariada \cite{cao2020delafo}. Além disso, os modelos podem ser combinados com métodos de otimização para a seleção de carteiras de investimentos \cite{du2022mean}.

        \ipar A aplicação de modelos de redes neurais com intuito de selecionar ativos para carteiras de investimentos é uma área que tem atraído a atenção de pesquisadores. O índice Sharpe tem sido utilizado em estudos como a função objetivo para a otimização de carteiras de investimentos com modelos de redes neurais \cite{tran2023optimizing}. Além disso, estuda-se a aplicação de redes neurais para predição do maior índice Sharpe no futuro \cite{vukovic2020neural}.

        \ipar Em geral, as redes neurais recorrentes (\acrshort{RNN}, do inglês \textit{Recurrent Neural Network}) têm se mostrado eficientes para previsão de séries temporais de ações \cite{wang2020portfolio}, em especial uma variante, a \acrshort{RNN} memória de longo prazo com curto prazo (\acrshort{LSTM}, do inglês \textit{Long Short-Term Memory}). Estes modelos de redes neurais utilizam uma camada de memória que permite a rede aprender dependências de longo prazo, e são capazes de capturar padrões de séries temporais não lineares para realizar previsões \cite{hochreiter1997long}.

        \ipar Há a possibilidade de combinar estruturas de redes neurais entre si para obter melhores resultados. O mecanismo de atenção tem sido efetivamente aplicado na área de processamento de linguagem natural, utilizado para melhorar o desempenho de modelos de redes neurais para previsão de séries temporais. Como exemplo, \citeonline{sun2022deep} desenvolveu um modelo para as séries temporais que combina uma rede neural convolucional com \textit{transformer} (rede neural baseada no mecanismo de atenção) para otimização de uma carteira de investimentos com objetivo de maximização do índice Sharpe.

        \ipar Em modelos de otimização há restrições e parâmetros que podem ser adicionados para tornar o problema próximo ao real. \citeonline{mulvey2020optimizing} propõe um modelo de otimização de carteira de investimentos com redes neurais que considera a taxa de transação. 
        
        \ipar A aplicação de restrições como a taxa de transação, limitação de capital, e restrições como a de cardinalidade, que limita o número de ativos na carteira tornam o problema próximo da realidade, contudo aumentam a complexidade de resolução \cite{aithal2023real}. Uma abordagem para resolução de problemas complexos é a utilização de métodos heurísticos, que são métodos de busca que não garantem a solução ótima, mas são capazes de encontrar soluções próximas da ótima em um tempo computacionalmente viável \cite{milhomem2020analysis}.

        \ipar Os estudos sobre otimização de carteiras de investimentos pelo índice Sharpe têm apresentado somente uma restrição real, o efeito do custo de transação. Em maioria negligenciam questões como lote padrão, a aversão ao risco e outras restrições afetam a decisão sobre a alocação de ativos. Estes parâmetros seguem as regulações e práticas exercidas no mercado em que se inserem.

        \ipar O índice Bovespa (\acrshort{IBOVESPA}) é o principal indicador de desempenho das ações do mercado de capitais brasileiro e referência para investidores, reunindo as empresas mais importantes do mercado nacional. A composição do ativo é construída com base nos seguintes critérios: quanto ao volume financeiro no mercado a vista, quanto índice de negociabilidade e presença no pregão, e quanto ao valor do ativo \cite{B32023}. 

        \ipar Portanto, este trabalho propõe a combinação dos métodos de otimização de carteira de investimentos com a aplicação de redes neurais para predição da carteira com melhor índice Sharpe, restrito a componentes do \acrshort{IBOVESPA}. Desta forma, busca-se construir carteiras de investimentos com base em modelos de risco e retorno por otimização do índice Sharpe com parâmetros reais. Avalia-se em seguida estruturas de redes neurais para seleção da melhor carteira de investimentos  para o próximo período.

    \end{subsection}

    \begin{subsection}{JUSTIFICATIVA}
        
        \ipar A teoria moderna de portfólio de Markowitz continua uma base amplamente utilizada para seleção de carteiras de investimentos. Gestores de fundos de investimentos e investidores individuais utilizam a teoria de Markowitz para selecionar carteiras de investimentos com base em modelos de risco e retorno. O processo de seleção de carteiras passa por diversas etapas, como a definição de um universo de ativos, a definição de um modelo de risco e retorno, a aplicação de um método de otimização para seleção de carteiras de investimentos, e a avaliação de desempenho da carteira de investimentos.

        \ipar Este processo é complexo e demanda tempo e conhecimento do investidor. A aplicação de modelos de redes neurais para previsão de índice Sharpe pode auxiliar o investidor a tomar decisões de alocação de ativos, e a prever o desempenho de sua carteira de investimentos. 

        \ipar Logo, a contribuição deste trabalho é o desenvolvimento de uma ferramenta de seleção de carteiras de investimentos com a construção de um fluxo de processamento que permita o investidor obter uma carteira de investimento para a sua realidade, com base em um modelo de risco e retorno, e com a aplicação de redes neurais para previsão de índice Sharpe. 

    \end{subsection}

    \begin{subsection}{OBJETIVOS}

        \begin{subsubsection}{Objetivo Geral}

            \ipar O objetivo principal deste trabalho é desenvolver e analisar uma estrutura que aplica redes neurais para seleção da melhor carteira de investimentos no período seguinte com base em carteiras otimizadas com índice Sharpe.

        \end{subsubsection}

        \begin{subsubsection}{Objetivos Específicos}
            \begin{enumerate}
                \item Realizar uma revisão sistemática da literatura sobre seleção de carteiras de investimentos, com base em modelos de risco e retorno, e com base em redes neurais.
                \item Construir ferramenta para coleta de dados históricos de ativos financeiros e dados econômicos de forma estruturada e com qualidade de dados.
                \item Desenvolver um modelo de alocação de ativos baseado no cálculo do índice Sharpe e com base em modelos de risco e retorno incluindo parâmetros reais.
                \item Elaborar estruturas de redes neurais para previsão de índice Sharpe e seleção da melhor carteira de investimentos.
                \item Analisar o desempenho da seleção de carteiras de investimentos aplicados a este fluxo de processamento de duas etapas. 
            \end{enumerate}

        \end{subsubsection}

    \end{subsection}

    
    \begin{subsection}{LIMITAÇÕES DA PESQUISA}
    
        \ipar O índice Sharpe é uma ferramenta amplamente utilizada para avaliar desempenho de carteiras de investimentos. Este estudo não busca avaliar a previsão de valor do índice Sharpe para uma carteira específica. O objetivo é utilizar a previsão de índice Sharpe para selecionar a melhor carteira de investimentos dentre um conjunto de carteiras de investimentos que aplicam o índice Sharpe como critério de seleção de ativos.

        \ipar O desenvolvimento deste estudo não contempla financiamento para aquisição de dados de mercado. Portanto, a coleta de dados de mercado é limitada a fontes gratuitas disponíveis na internet. O que tornou necessário o desenvolvimento de uma ferramenta de coleta de dados de mercado de forma estruturada e com qualidade de dados.

        \ipar O mercado financeiro é um ambiente dinâmico e complexo. Além disso, há diversos produtos financeiros com diferentes características. Portanto, este estudo não contempla a avaliação de todos os produtos financeiros disponíveis no mercado, se restringindo a avaliação de ativos financeiros de renda variável que compõem o índice Bovespa.

        \ipar Corretoras intermedeiam as transações de compra e venda de ativos financeiros. Portanto, há custos de transação para compra e venda de ativos financeiros, além de custos de custódia. Este estudo se limita a dados obtidos de somente uma corretora para referência.

    \end{subsection}

    \begin{subsection}{ESTRUTURA DE PESQUISA}
    
        \ipar Nesta seção de introdução, foi apresentado o contexto, a justificativa, os objetivos e as limitações da pesquisa. O restante deste trabalho está organizado da seguintes seções. Na seção 2 é apresentado o referencial teórico, com uma avaliação sistemática da literatura, em que avalia os principais textos que fundamentam os estudos recentes, além de identificar nos estudos recentes quais os métodos utilizados para o problema de otimização de carteiras considerando a utilização do índice Sharpe. Essa seção explora a fundamentação dos conceitos e métodos utilizados neste estudo. 
        
        \ipar Na seção 3 é apresentada a metodologia de pesquisa, com a descrição dos dados utilizados, a descrição do modelo de otimização de carteiras de investimentos, e descrição das estruturas de redes neurais para previsão de índice Sharpe, e a metodologia aplicada para validação dos modelos.
        
        \ipar Na seção 4 são apresentados os resultados obtidos com a aplicação dos modelos de otimização de carteiras de investimentos e de redes neurais para previsão de índice Sharpe. Na seção 5 são apresentadas as conclusões e sugestões para trabalhos futuros. Na seção seguinte é apresentada as referências bibliográficas utilizadas neste estudo. 
        
        \ipar Apêndices são apresentados ao final deste trabalho, contendo as peças de códigos desenvolvidos neste trabalho, e a produção científica gerada durante o desenvolvimento deste estudo.

    \end{subsection}

\end{section}

\pagebreak